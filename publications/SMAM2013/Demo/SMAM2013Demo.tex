\documentclass{llncs}
\usepackage{graphicx}
\usepackage{times}
\usepackage{llncsdoc}
\usepackage{color}
\usepackage{url}
%\usepackage{amssymb}
\usepackage{rotating}

\begin{document}

\title{Mapping and Exposing MusicBrainz with Linked Data}

\author{Peter Haase\inst{1} and Barry Norton\inst{2} and Juan Sequeda\inst{3}}

\institute{fluid Operations AG, Walldorf, Germany \email{peter.haase@fluidops.com}
\and Ontotext AD, Sofia, Bulgaria \email{barry.norton@ontotext.com}
\and University of Texas, Austin, United States \email{jsequeda@cs.utexas.edu}}

\maketitle

\begin{abstract}
In this demo we shall show how MusicBrainz data is mapped from its
relational source into RDF via R2RML, and how it is interlinked with
other sources such as DBpedia. We shall show, furthermore, how a
music-oriented application can be quickly built over such cross-set
Linked Data, including visualisation, using the Information
Workbench. Other tools that will be demonstrated include Ultrawrap, as
an R2RML engine, and OWLIM, and a triplestore.
\end{abstract}

\section{Introduction}\label{sec:Introduction}
A vision of MusicBrainz as a rich Semantic Web resource is a
long-standing one~\cite{DBLP:journals/expert/Swartz02}. In reality,
however, MusicBrainz has had a chequered history with Semantic
Technologies. In this demo we'll show how the latest approach to
bringing together Linked Data technologies and principles with the
MusicBrainz dataset enable relatively much easier application
construction with latest generation tools.

\section{Conclusion}\label{sec:Conclusion}

\bibliographystyle{splncs03}
\bibliography{../../LinkedBrainz}

\end{document}
